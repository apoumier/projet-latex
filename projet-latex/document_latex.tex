\documentclass{article}
\usepackage[utf8]{inputenc}
\usepackage{graphicx}
\usepackage{listings}
\usepackage{tabularx}
\usepackage{array}
\usepackage{color}

\title{\Huge{\textcolor{blue}{\textbf{Aires et périmètres des figures de base}}}}
\author{\Huge{Poumier Antonin}}
\date{\Huge{\today}}

\usepackage{fancyhdr}
\pagestyle{fancy}
\fancyhead[L]{\LaTeX}
\fancyhead[R]{Université Paris 8}
\fancyfoot[C]{Outils Collaboratifs}
\fancyfoot[R]{\today}
\begin{document}

\maketitle
\begin{figure}[b]
	\centering
	\includegraphics[scale=0.5]{paris8.png}
\end{figure}
\newpage
\tableofcontents
\listoffigures
\listoftables
\newpage
\large{Nous allons voir les différentes formules pour calculer l'aire et le périmètre des figures suivantes.}
\begin{enumerate}
	\item Les carrés
	\item Les rectangles
	\item Les triangles
	\item Les cercles/disques 
\end{enumerate}
\section{\large{Aires}}
	\subsection{\large{Aires de ces figures}} 
		\large{Pour calculer l'aire de ces figures, il existe plusieurs formules.}
		\begin{equation}
			A_c = c \times c 
		\end{equation}
		\begin{equation}
		A_r = L \times l 
		\end{equation}
		\begin{equation}
		A_t = \frac{base \times hauteur}{2}
		\end{equation}
		\begin{equation}
		A_d = \Pi\,\times\,R\,\times\,R\
		\label{4}
		\end{equation}
		Exemple pour un disque de rayon 4 cm on a :
		\begin{equation}
			\Pi\,\times\,4\,\times\,4\,=\,16\,\times\,\Pi\,cm^{2}\approx\,50,24\,cm^{2}
		\end{equation}
		Attention, il faut arrondir les valeurs de la formule \ref{4} ! \\
	\subsection{\large{Représentation de quelques figures}}
	\begin{figure}[h]
		\centering
		\includegraphics[scale=0.5]{figures.png}
		\caption{\large{Carré et rectangle}}
	\end{figure}


\section{\large{Périmètre}}
\subsection{\large{Périmètre de ces figures}}
		\large{Pour calculer le périmètre de ces figures, il existe plusieurs formules.}
\begin{equation}
	A_c = 4 \times c 
\end{equation}
\begin{equation}
	A_r = (2 \times L) + (2 \times l) 
\end{equation}
\begin{equation}
	A_t = L1 + L2 + L3
\end{equation}
\begin{equation}
	A_d = 2 \times\Pi\times R
	\label{9}
\end{equation}
Exemple pour un cercle de rayon 4 cm on a :
\begin{equation}
	2 \times \Pi \times 4 \approx 25,13 \\
\end{equation} 
Attention, il faut arrondir les valeurs de la formule \ref{9} ! \\
\subsection{\large{Représentation d'autres figures}}
	\begin{figure}[h]
	\centering
	\includegraphics[scale=0.7]{figures2.png}
	\caption{\large{Triangle et cercle}}
\end{figure}
    \begin{center}
	 C'est Archimède\cite{fleurs}, fils de l'astronome Phidias, mathématicien grec vivant à Syracuse\footnote{Ville italienne de la côte ionienne, en Sicile, faisant partie de la Grande Grèce à l'époque.}(287 av. J.-C./212 av. J.-C.)  qui est la première personne à démontrer vers 250 avant J. -C les formules du cercle et la constante Pi qui interviennent dans le calcul de l'aire et du périmètre.  
\end{center}
\begin{thebibliography}{1}
	\bibitem{fleurs} https://www.histoiredumonde.net/Archimede.html
\end{thebibliography}
\section{\large{Algorithme en Python}}
\begin{table}[h]
	\begin{tabular}{l|c|r}
		Carré & Rectangle \\ \hline
\begin{lstlisting}[language=python]
def perimetrecarre(c):
	p = 4 * c
	return p
\end{lstlisting}&
\begin{lstlisting}[language=python]
def perimetrerectangle(L,l):
	p = (2 * L) + (2 * l)
	return p
\end{lstlisting}
	\end{tabular}
	\caption{Carré et Rectangle}
\end{table}
\begin{table}[h]
	\begin{tabular}{l|c|r}
		Triangle & Cercle \\ \hline
\begin{lstlisting}[language=python]
def perimetretriangle(L1,L2,L3):
	p = L1 + L2 + L3
	return p
\end{lstlisting}&
\begin{lstlisting}[language=python]
from math import *
def perimetrecercle(r):
	p = 2 * pi * r
	return p
\end{lstlisting}
	\end{tabular}
\caption{Triangle et Cercle}
\end{table}
\end{document}


